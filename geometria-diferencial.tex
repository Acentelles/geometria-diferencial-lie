\documentclass[12pt]{extarticle}
%Some packages I commonly use.
\usepackage[english]{babel}
\usepackage{graphicx}
\usepackage{framed}
\usepackage[normalem]{ulem}
\usepackage{amsmath}
\usepackage{amsthm}
\usepackage{amssymb}
\usepackage{amsfonts}
\usepackage{enumerate}
\usepackage[utf8]{inputenc}
\usepackage[top=1 in,bottom=1in, left=1 in, right=1 in]{geometry}

%A bunch of definitions that make my life easier
\newcommand{\matlab}{{\sc Matlab} }
\newcommand{\cvec}[1]{{\mathbf #1}}
\newcommand{\rvec}[1]{\vec{\mathbf #1}}

\newcommand{\ihat}{\hat{\textbf{\i}}}
\newcommand{\jhat}{\hat{\textbf{\j}}}
\newcommand{\khat}{\hat{\textbf{k}}}
\newcommand{\minor}{{\rm minor}}
\newcommand{\trace}{{\rm trace}}
\newcommand{\spn}{{\rm Span}}
\newcommand{\rem}{{\rm rem}}
\newcommand{\ran}{{\rm range}}
\newcommand{\range}{{\rm range}}
\newcommand{\mdiv}{{\rm div}}
\newcommand{\proj}{{\rm proj}}
\newcommand{\R}{\mathbb{R}}
\newcommand{\N}{\mathbb{N}}
\newcommand{\Q}{\mathbb{Q}}
\newcommand{\C}{\mathbb{C}}
\newcommand{\Z}{\mathbb{Z}}
\newcommand{\T}{\mathbb{T}}
\newcommand{\g}{\mathfrak{g}}
\newcommand{\h}{\mathfrak{h}}
\newcommand{\gl}{\mathfrak{g}\mathfrak{l}}
\newcommand{\<}{\langle}
\renewcommand{\>}{\rangle}
\renewcommand{\emptyset}{\varnothing}
\newcommand{\attn}[1]{\textbf{#1}}
\theoremstyle{definition}
\newtheorem{theorem}{Theorem}
\newtheorem{corollary}{Corollary}
\newtheorem*{definition}{Definition}
\newtheorem*{example}{Example}
\newtheorem*{note}{Note}
\newtheorem{exercise}{Exercise}
\newcommand{\bproof}{\bigskip {\bf Proof. }}
\newcommand{\eproof}{\hfill\qedsymbol}
\newcommand{\Disp}{\displaystyle}
\newcommand{\qe}{\hfill\(\bigtriangledown\)}
\setlength{\columnseprule}{1 pt}


\title{La presencia de Lie en la geometria diferencial}
\author{Alberto Centelles}
\date{}

\begin{document}

\maketitle

\section{Variedades diferenciables}
Las variedades diferenciables son espacios que "localmente se parecen" a un espacio
euclídeo $\mathbb{R}^n$ y sobre el que uno puede aplicar conceptos del analisis matematico.
Las variedades mas simples son las variedades topológicas, que son
espacios topologicos con ciertas propiedades que expresan con exactitud lo que
refererimos como "localmente se parecen" a $\mathbb{R}^n$.
La nocion de "suave" es intuitiva cuando nos referimos a variedades que son
subconjuntos de $\mathbb{R}^n$. Por ejemplo, llamamos a una curva "suave" si es
posible definir una recta tangente que varia de punto a punto de forma continua
y llamamos a una superficie "suave" si existe un plano tangente que tambien
varia de forma continua.
Para variedades diferenciables mas complejas resulta innecesario embeberlas como
subconjuntos de un espacio euclideo ambiente. Por esa razon, es necesario pensar
en variedades diferenciables como espacios topologicos abstractos sin la necesidad de
que sean subconjuntos de espacios mas grandes.
No es dificil ver que no hay forma de definir una propiedad puramente topologica
que nos sirva como criterio de "suavidad" porque esta propiedad no puede ser
invariantes bajo homeomorfismos. Por ejemplo, un cuadrado y un circulo en el
plano son espacios topologicos homeomorfos pero el circulo es "suave" mientras
que el cuadrado no lo es. Por tanto, las variedades topologicas no son
suficientes; debemos pensar en una variedad diferenciable como un conjunto con dos
estructuras adicionales: una topologia y una estructura suave.

\subsection{Estructura topologica}
Dado un espacio topologico $M$, decimos que $M$ es una variedad topologica de
dimension $n$ si tiene las siguientes propiedades:

- $M$ es un espacio de Hausdorff: para cada par de puntos distintos $p, q \in M$
existen conjuntos abiertos disjuntos $U, V \subseteq M$ tales que $p \in U$ y
$q \in V$
- $M$ es segundo contable: existe una base contable para la topologia de $M$.
- $M$ es localmente euclideo de dimension $n$: cada punto de $M$ tiene un
entorno que es homeomorfico a un conjunto abierto de $\mathbb{R}^n$.

% * La cohomologia de De Rham: La dimension de una variedad topologica no vacia es una
% invariante topologica.

Como ejemplo de una variedad topologica podemos tomar un $n$-toro para $n$
positivo, es decir, el espacio producto $\mathbb{T}^n = \mathbb{S}^1 \times ...
\times \mathbb{S}^1$. Sabemos que toda $n$-esfera, incluida la circunferencia
$\mathbb{S}^1$, es una variedad topologica. Es un espacio de Hausdorff pues
cualquier producto finito de espacios de Hausdorff es un espacio de Hausdorff.
De la misma forma deducimos que es un espacio segundo contable. Veamos que es
localmente euclideo: para todo punto $(p_1, ..., p_k) \in \mathbb{S}^1 \times
... \times \mathbb{S}^1$, podemos elegir una carta de coordenadas ($U_i,
\varphi_i$) para cada $\mathbb{S}^1$ con $p_i \in U_i$. La aplicacion producto
$\varphi_1 \times ... \times \varphi_k : U_1 \times ... \times U_k \rightarrow
\mathbb{R}^{n_1 + ... +n_k}$
es un homeomorfismo sobre su imagen, que es un conjunto abierto de
$\mathbb{R}^{n_1 + ... + n_k}$. Por tanto, el $n$-torus $\mathbb{T}^n$ es una
variedad topologica.

\subsection{Estructura suave}
Con la estructura topologica expuesta en el apartado anterior, se pueden
estudiar propiedades topologicas de las variedades como la compacidad,
conectividad y cualquier problema de clasificacion de variedades topologicamente
equivalentes. Sin embargo, la teoria de las variedades topologicas no hace
mencion del analisis matematico. Esto se debe a que las derivadas de funciones
sobre variedades no son invariantes bajo homeomorfismo.

Para dar sentido a las derivadas de aplicaciones entre variedades es necesario
tratar con variedades diferenciables. Dados $U \subset \R^n$ y $V \subset \R^m$,
una funcion $F: U \rightarrow V$ se dice que que es suave ($C^{\infty}$ o
infinitamente diferenciable) si cada una de sus funciones componente tiene
derivadas parciales continuas de todo orden. Si ademas $F$ es biyectiva y tiene
una aplicacion inversa suave, se le llama difeomorfismo. Un difeomorfismo es, en
particular, un homeomorfismo.

El reto es ver como podemos extender la estructura de una variedad topologica
para discernir que aplicaciones son diferenciables ($C^{\infty}$). Dada una
$n$-variedad topologica $M$, cada punto $p \in M$ es el dominio de una aplicacion
$\varphi: U \rightarrow V \subseteq \R$. Podemos decir que $f : M \rightarrow \R$ es diferenciable
si y solo si la composicion $f \circ \varphi^{-1}: V \rightarrow \R$ es suave en
el sentido ordinario del calculo. Sin embargo, esta definicion solo tiene
sentido si esta propiedad es independiente de la eleccion de la carta. Para
garantizar esta independencia, debemos de hablar de "cartas suaves" y considerar
la coleccion de todas las cartas suaves como una estructura adicional sobre $M$.

Si $(U, \varphi)$, $(V, \psi)$ son dos cartas tales que $U \cap V \neq
\emptyset$, a la composicion $\psi \circ \varphi^{-1}: \varphi(U \cap V)
\rightarrow \psi(U \cap V)$ se le denomina funcion de transicion de $\varphi$ a
$\psi$. Dos cartas $(U, \varphi)$ y $(V, \psi)$ son (diferenciablemente)
compatibles si la funcion de transicion $\psi \circ \varphi^{-1}$ es un
difeomorfismo o $U \cap V = \emptyset$.

Un atlas de $M$ es una coleccion de cartas cuyos dominios cubren $M$. Un atlas $A$
es suave si dos cartas cualesquiera en $A$ son diferenciablemente compatibles.
Dado que, en general, existen varios atlas que proporcionan la misma
estructura diferenciable, se elige por conveniencia el atlas maximal.

\subsection{Variedades diferenciables}
Si $M$ es una variedad topologica, una estructura suave sobre $M$ es una atlas
maximal suave. Una variedad diferenciable es un par $(M, A)$.

El espacio de matrices sirve como ejemplo de variedad diferenciable.
Sea $M(m \times n, \R)$ el conjunto de matrices $m \times n$ de componentes
reales. Debido a que es un espacio vectorial real de dimension $mn$ bajo la
adicion de matrices y multiplicacion por escalares, $M(n \times n, \R)$ es una
variedad $mn$-diferenciable. De la misma forma, el espacio $M(m \times n, \C)$
de $m \times n$ matrices complejas es un espacio vectorial de dimension $2mn$
sobre $\R$ y, por tanto, una variedad $2mn$-dimensional.

El ejemplo del espacio de matrices ira apareciendo en cada una de las siguientes
secciones para ilustrar los distintos conceptos de Lie.

\section{Grupos de Lie}
Estamos ahora en disposicion de entender los grupos de Lie. Un grupo de Lie es
una variedad diferenciable $G$ (sin frontera) que es tambien un grupo en el
sentido algebraico, con la propiedad que la aplicacion multiplicacion $m : G
\times G \rightarrow G$ y la aplicacion inversion $i : G \rightarrow G$, dadas
por $m(g, h)=gh$ y $i(g) = g^{-1}$ son ambas suaves. Un grupo de Lie es, en
particular, un grupo topologico (esto es, un espacio topologico con una
estructura de grupo tal que las aplicaciones multiplicacion e inversion son
continuas).

Destacan dos aplicaciones sobre los grupos de Lie: la traslacion por la
izquierda $L_g(h) = gh$ y la traslacion por la derecha $R_g(h)=hg$, donde $G$ es
un grupo de Lie y $g, h \in G$.

Un ejemplo de grupo de Lie es el grupo general lineal $GL(n, \R)$ definido como
el conjunto de matrices $n \times n$ invertibles con componentes reales. Se
trata de un grupo bajo la multiplicacion de matrices y es un subgrupo del
espacio vectorial $M(n, \R)$, mostrado en el ejemplo anterior. La multiplicacion
es suave porque las componentes de la matriz producto $AB$ son polinomios en las
entradas de $A$ y $B$. La inversion es suave por la regla de Cramer.

\subsection{Homomorfismos de grupos de Lie}
Si $G$ y $H$ son grupos de Lie, definimos homomorfismo de grupos de Lie de $G$ a
$H$ como una aplicacion suave $F: G \rightarrow H$ que es tambien un
homomorfismo de grupos. Se dice que es isomorfismo de grupos de Lie si es
tambien un difeomorfismo, lo que implica que tiene una inversa que es tambien un
homomorfismo de grupos de Lie. En este caso decimos que $G$ y $H$ son grupos de
Lie isomorficos.

Un ejemplo de homomorfismo de grupos de Lie es el de la funcion determinante
$det: GL(n,\R) \rightarrow \R^*$. Es una aplicacion suave porque $det A$ es un
polinomio en las entradas de las matrices de $A$. Es un homomorfismo de grupos
de Lie porque $det(AB) = (det A)(det B)$. De la misma forma, $det: GL(n, \C)
\rightarrow \C^*$ es un homomorfismo de grupos de Lie.

\section{Subgrupos de Lie}
Para entender los subgrupos de Lie es necesario estudiar con especial atencion
cierto tipo de aplicaciones diferenciales: aquellas cuyas diferenciales tienen
rango constante.

\subsection{La diferencial de una aplicacion}
Para relacionar los espacios tangentes abstractos definidos en variedades con
espacios tangentes geometricos en $\R^n$, debemos explorar la forma en la que
las aplicaciones diferenciales modifican los vectores tangentes. En el caso de
una aplicacion diferenciable entre espacios euclideos, la derivada total de una
aplicacion en un punto (representada por su matriz jacobiana) es una aplicacion
lineal que representa "la mejor aproximacion" de una aplicacion en el entorno de
un punto. En el caso de las variedades diferenciales, no tiene sentido hablar de
una aplicacion lineal entre
variedades diferenciales. En su lugar, tiene sentido hablar de una aplicacion
lineal entre espacios tangentes.

Si $M$ y $N$ son variedades diferenciales y $F: M \rightarrow N$ es una
aplicacion diferenciable, para cada $p \in M$ decimos que la diferencial de $F$ en
$p$ es una aplicacion $dF_p:T_pM \rightarrow T_{F(p)N}$ definida como
$dF_p(v)(f) = v(f \circ F)$, donde $v \in T_pM$ y $dF_p(v)$ es la derivada en
$F(p)$ que actua sobre $f \in C^{\infty}$.

\subsection{Inmersiones, encajes y sumersiones}
El rango es la unica propiedad de la diferencial que puede ser definida
con independencia de la eleccion de las bases. Dada una aplicacion diferenciable
$F:M \rightarrow N$ y un punto $p \in M$, el rango de $F$ en $p$ se define como
el rango de la aplicacion lineal $dF_p:T_pM \rightarrow T_{F(p)N}$, esto es, la
dimension de $Im (dF_p) \subseteq T_{F_p}N$. Si $F$ tiene el mismo rango en cada
punto se dice que tiene rango constante.

El valor del rango de una aplicacion lineal nunca puede ser mayor que la
dimension del dominio o el codominio, luego esta acotado. Si el rango de la
diferencial $dF_p$ es igual a su cota superior, se dice que $F$ tiene rango
completo en $p$. Si $F$ tiene rango completo en todo punto de $M$, entonces
$F$ tiene rango completo.

Las aplicaciones de rango constante mas importante son las de rango completo.
Una aplicacion diferenciable $F: M \rightarrow N$ se dice que es una sumersion
suave si su diferencial es sobreyectiva en cada punto (o equivalentemente, si
si el rango de $F$ es igual $dim N$). Se dice que es una inmersion suave si su
diferencial es inyectiva en cada punto (equivalentemente, $rango F = dim M$).

De todas las inmersiones, los encajes son de particular importancia. Un encaje
suave es una inmersion suave que ademas es un encaje topologico, esto es, un
homeomorfismo sobre sus imagenes $F(M) \subseteq N$ en la topologia inducida. Se
trata de una aplicacion que es a la vez un encaje topologico y una inmersion
suave.

Existen varias proposiciones y teoremas que analizan las propiedades de las
aplicaciones suaves entre variedades diferenciables, entre ellos el teorema de
la funcion inversa para variedades diferenciables o su corolario, el teorema del
rango constante.
Ponemos a continuacion una de estas proposiciones como ilustracion:

Sea $F: M \rightarrow N$ una aplicacion suave y $p \in M$. Si $dF_p$ es
sobreyectiva, entonces $p$ tiene un entorno $U$ tal que $F\mid_U$ es una
sumersion. Si $dF_p$ es inyectiva, entonces $p$ tiene un entorno $U$ tal que
$F\mid_U$ es una inmersion.

Estas juegan un papel destacado en la teoria de las variedades diferenciales y,
como veremos, en los subgrupos de Lie. Como ejemplo de sumersion destacamos la
proyeccion $\pi : TM \rightarrow M$, donde $M$ es una variedad diferencial y
$TM$ su fibrado tangente. Para verificar que es una sumersion, basta ver que con
respecto a cualquier sistema de coordenadas local $(x^i)$ en un abierto $U
\subseteq M$ y las correspondientes coordenadas naturales $(x^i, v^i)$ en
$\pi^{-1}(U) \subseteq TM$, la representacion de las coordenads de $\pi$ es
$\pi(x, v) = x$.

Un ejemplo de inmersion suave de $\R^2$ en $\R^3$ es el dado por la aplicacion
suave $X: \R^2 \rightarrow \R^3$, $X(u, v) = ((2 + cos2 \pi u) cos2\pi v, (2 +
cos2\pi u)sen2\pi v, sen2\pi u)$, cuya imagen es un toro de revolucion en $\R^3$
obtenido a partir del circulo $(y - 2)^2 + z^2 = 1$. Otro ejemplo destacado de
inmersion es el de una curva suave $\gamma : J \rightarrow M$: se dice que
$\gamma$ es una inmersion suave si y solo si $\gamma'(t) \neq 0$ para todo $t
\in J$.

Un ejemplo ilustrativo de encaje suave es el de la aplicacion inclusion $X: U
\rightarrow M$, donde $U$ es una subvariedad abierta de una variedad diferencial
$M$. Otro ejemplo de encaje suave es el dado por cada uno de las aplicaciones
$\iota_j : M_j \rightarrow M_1 \times ... \times M_k$, $\iota_j(q) = (p_1, ...,,
p_{j-1}, q, p_{p+1},..., p_k)$. En particular, la aplicacion inclusion de $\R^n$
a $\R^{n+k}$ que manda $(x_1,..., x^n)$ a  $(x^1, ..., x^n, 0,..., 0)$ es un
encaje suave.


Otro ejemplo caracteristico que ilustra la diferencia entre encaje e inmersion
es el de la lemniscata de Bernoulli. Esta es la
curva en $\R^2$ cuya trayectoria obedece la ecuacion polinomial de cuarto
grado:$(x^2+y^2)^2 = 2a^2(x^2-y^2)$,donde $a>0$ es un parametro. Esta curva se puede
parametrizar por las funciones $x(t):=\dfrac{a\sqrt{2} cos(t)}{1+sen^2t}$,
$y(t) = \dfrac{a\sqrt{2}cost(t) sen(t)}{1+sen^2(t)}$,
En otras palabras, la curva $\gamma: \R \rightarrow \R^2$ dada por $\gamma(t) =
(x(t),y(t))$ recorre la trayectoria dada (infinitas veces); para obtener un solo
recorrido hay que restringir $\gamma$ a un intervalo de longitud $2\pi$, por
ejemplo $[0, 2\pi)$. Esta parametrizacion es regular porque $\gamma'(t) \neq (0,
0)$
para todo $t \in \R$. Entonces el vector tangente $\gamma_*(t): v \rightarrow
\gamma'(t)v$ no se anula en el punto $\gamma(t)$ de la curva. Luego cada
$\gamma_*(t)$ es inyectiva, asi que $\gamma$ es una inmersion de $\R$ en
$\R^2$. Sin embargo, la curva $\gamma$ no es un encaje. Aun cuando se considera
la restriccion de $\gamma$ a una funcion suave del intervalo abierto $(0, 2\pi)$
al abierto $\R^2/(a\sqrt{2}, 0)$ en el plano, esta tampoco es un
encaje porque no es inyectiva, como evidencia el punto doble $(0,
0)=\gamma(\pi/2)=\gamma(3\pi/2)$.

Como contraejemplo, la aplicacion diferencial $\gamma : \R \rightarrow \R^2$ dada por
$\gamma(t) = (t^3, 0)$ es una aplicacion diferencial y una encaje topologico,
pero no es un encaje suave porque $\gamma'(0)=0$.

Los encajes diferenciales dan lugar a las subvariedades encajadas en una
variedad diferencial $M$. Una subvariedad encajada en $M$ es un subconjunto $S
\subseteq M$ que es una variedad en la topologia inducida, dotado de una
estructura diferencial respecto a la cual la aplicacion inclusion $S \rightarrow
M$ es un encaje suave.

\subsection{Subgrupos de Lie}

Dado un grupo de Lie $G$, $H \subseteq G$ es un subgrupo de Lie de
$G$ si es un subgrupo de $G$ y esta dotado de una topologia y una estructura
suave que lo convierten en un grupo de Lie y en una subvariedad diferencial
inmersa en $G$. Todo subgrupo que ademas es una subvariedad encajada
en $G$ es automaticamente un subgrupo de Lie. El reciproco no es siempre cierto.
Por ejemplo, sea $H \subseteq \T^2$ una subvariedad densa del toro que es imagen
de la inmersion $\gamma : \R \rightarrow \T^2$ definida por $\gamma(t)=(e^{2 \pi
it}, e^{e \pi i \alpha t})$. Es una inmersion suave porque $\gamma'(t)$ nunca se
anula. Tambien es inyectiva, pues $\gamma(t_1) = \gamma(t_2)$ implica que ambos
$t_1 - t_2$ y $\alpha t_1 - \alpha t_2$ son enteros, lo cual es imposible. Es
facil ver que $\gamma$ es un homomorfismo de grupos inyectivo y, por tanto, $H$
es un subgrupo de Lie inmerso en $\T^2$.

Siguiendo la linea de ejemplos sobre matrices, el grupo especial lineal $SL(n,
\R)$ de $n \times n$ matrices reales con determinante igual a 1 es un ejemplo de
subgrupo de Lie de $GL(n, \R)$, ya que $SL(n, \R)$ es el
nucleo del homomorfismo de grupos de Lie $det: GL(n, \R) \rightarrow \R^*$ y
$det$ es una funcion sobreyectiva, luego es una sumersion suave por el teorema
del rango global. La dimension de $SL(n, \R)$ es $n^2 - 1$.

\section{Algebra de Lie}
\subsection{Campos vectoriales}
Para entender el algebra de Lie es necesario familiarizarse con el concepto de
campo vectorial. En el espacio euclideo $\R^n$ puede visualizarse poniendo una
"flecha" en cada punto de un abierto $U \subseteq \R^n$. Esta visualizacion no
es tan util cuando extendemos el concepto a una variedad diferencial abstracta.
Debemos pensar en un campo vectorial en una variedad diferencial $M$
como una aplicacion continua de $M$ a su fibrado tangente $TM$, es decir, una
aplicacion que asigna a cada punto $p \in M$ su vector tangente $X_p \in T_pM$.
Formalmente, es una seccion de la aplicacion $\pi: TM \rightarrow M$, esto es,
una aplicacion $X: M \rightarrow TM$ con la propiedad $\pi \circ X = Id_M$, o
equivalentemente, $X_p \in T_pM$ para cada $p \in M$.

Una de las propiedades mas importantes de los campos vectoriales es que definen
operadores en el espacio de las funciones reales suaves. Si $X \in
\mathfrak{X}(M)$ y $f$ es una funcion real suave definida en un abierto $U \in
M$, obtenemos una nueva funcion $Xf : U \rightarrow \R$, definida por $(Xf)(p) =
X_p f$. Dado que la accion de un vector tangente en una funcion esta determinada
por los valores de una funcion un entorno arbitrariamente pequeño, se sigue que
$Xf$ esta localmente determinado. En particular, para todo abierto $V \subseteq
U$, $(Xf)|_V = X (f|_V)$.

No todos los campos vectoriales son suaves (ni continuos), pero nosotros nos
centramos en los campos vectoriales suaves. Un campo vectorial suave $X \in
\mathfrak{X}(M)$ define una aplicacion de $C^{\infty}$ a si misma, que es lineal
sobre $\R$. Ademas, podemos trasladar la regla del producto de vectores tangente
a la regla del producto de campos vectoriales $X(fg) = f Xg + gXf$. En general,
una aplicacion de este tipo, $C^{\infty}(M) \rightarrow C^{\infty}(M)$ se
llama derivacion si es lineal sobre $\R$ y satisface $X(fg) = f Xg + gXf$ para
todo $f, g \in C^\infty$. Por tanto, las derivaciones en $C^\infty$ se pueden
identificar con los campos vectoriales suaves.

Si $F:M\rightarrow N$ es una aplicacion diferencial y $X$ es un campo vectorial
en $M$, entonces para cada punto $p \in M$, obtenemos un vector $dF_p(X_p) \in
T_{F(p)}N$ evaluando la diferencial de $F$ sobre $X_p$. Sin embargo, esto en
general no define un campo vectorial en $N$. Por ejemplo, si $F$ no es
sobreyectiva, no hay forma de decidir que vector asignar a un punto $q \in N$
tal que $q \neq F(M)$. Si $F$ no es inyectiva, para algunos puntos de $N$ habra
varios vectores obtenidos al evaluar $dF$ sobre $X$ en diferentes puntos de $M$.

Cuando $dF(X)$ representa un campo vectorial $Y$ sobre $N$, decimos que $X$ e
$Y$ estan $F$-relacionados si y solo si para toda funcion real suave $f$
definida en un abierto de $N$, $X(f \circ F) = (Yf) \circ F$. Por ejemplo, dada la
aplicacion suave de $\R$ a $\R^2$ dada por $F(t) = (\cos t, \sin t)$, los campos
vectoriales $d/dt \in \mathfrak{X}(\R)$ e $Y \in \mathfrak{X}(\R^2)$ definido
por $Y=x\dfrac{\partial}{\partial y} - y\dfrac{\partial}{\partial x}$ estan
$F$-relacionados.

A veces no existe ningun campo vectorial en $N$ que esta $F$-relacionado con un
campo vectorial $X  \in \mathfrak{X}(M)$, dada una aplicacion $F : M \rightarrow
N$. Pero cuando $F$ es un difeomorfismo, se tiene que existe un campo vectorial
suave en $N$ y que ademas es unico. Este campo vectorial suave y unico se llama
\emph{pushforward} de $X$ por $F$ y se puede definir explicitamente por la formula
$(F_*X)_q = dF_{F^{-1}(q)}(X_{F^{-1}(q)})$. Ademas, si $F$ es un difeomorfismo,
podemos reescribir la $F$-relacion de dos campos vectoriales del parrafo
anterior por $((F_*X)f) \circ F = X(f \circ F)$.

El siguiente ejemplo muestra como calcular el \emph{pushforward} de un campo
vectorial: Sean $M = \{(x, y): y>0, x+y > 0\}$ y $N = \{(u, v): u>0, v>0\}$ dos
variedades diferenciales. Definimos $F: M \rightarrow N$ como $F(x, y) =
(x+y,x/y + 1)$, que es un difeomorfismo cuya inversa $F^{-1}(u,v)=(u-u/v, u/v)$
es tambien continua para $v>0$. El campo vectorial dado $X \in \mathfrak{X}(M)$
es $X_{(x,y)} = y^2 \dfrac{\partial}{\partial x} |_{(x, y)}$. La diferencial de
$F$ en el punto $(x,y) \in M$ se representa por su matriz jacobiana
\begin{equation*}
	DF(x,y) = \begin{pmatrix}
		1 & 1 \\
		\dfrac{1}{y} & -\dfrac{x}{y^2}
	\end{pmatrix}
\end{equation*}
luego $dF_{F^{-1}(u,v)}$ se representa por la matriz
\begin{equation*}
	DF(u-\dfrac{u}{v},\dfrac{u}{v}) = \begin{pmatrix}
		1 & 1 \\
		\dfrac{v}{u} & -\dfrac{v - v^2}{u}
	\end{pmatrix}
\end{equation*}
para todo $(u, v) \in N$, $X_{F^{-1}(u, v)} =
\dfrac{u^2}{v^2}\dfrac{\partial}{\partial x}|_{F^{-1}(u,v)}$ y aplicando la
formula $(F_*X)_q = dF_{F^{-1}(q)}(X_{F^{-1}(q)})$ con $q=(u,v)$ tenemos que
$(F_*X)(u,v) = \dfrac{u^2}{v^2}\dfrac{\partial}{\partial u}|_{(u, v)} +
\dfrac{u}{v}\dfrac{\partial}{\partial v}|_{(u,v)}$.

\subsection{Corchete de Lie}
No todas las combinaciones de campos vectoriales dan lugar a otro campo
vectorial. Uno podria pensar que dados dos campos vectoriales $X$ e
$Y$ en una variedad diferencial $M$ y dada una funcion $f: M \rightarrow \R$, la
funcion $YXf=Y(Xf)$ es un campo vectorial. Sin embargo, la operacion $f \mapsto
YXf$ no satisface en general la regla del producto y, por tanto, no puede ser un
campo vectorial. Por ejemplo, sean los campos vectoriales $X = \partial / \partial x$
e $Y = x \partial / \partial y$ en $\R^2$ y $f(x, y)=x, g(x, y)=y$. Entonces
$XY(fg) = 2x$ mientras que $fXYg + gXYf = x$, por lo que $XY$ no es una
derivacion en $C^{\infty}(\R^2)$.
El corchete de Lie, $[X, Y]: C^{\infty}(M) \rightarrow C^{\infty}(M)$, es una
manera especial de combinar dos campos vectoriales suaves para dar lugar a un campo
vectorial tambien suave. Esta definido por:
\begin{equation*}
[X, Y]f = XYf - YXf
\end{equation*}

El valor del campo vectorial $[X, Y]$ en un punto $p \in M$ es la derivacion de
$p$ dada por la formula $[X, Y]_p f = X_p(Yf) - Y_p(Xf)$. Desafortunadamente,
esta formula es de escasa utilidad para los calculos, pues requiere hallar los terminos
de las segundas derivadas de $f$ que siempre se cancelaran. En su lugar es mas
comun utilizar la formula de las coordenadas del corchete de Lie, que dice asi:

Sean $X, Y$ campos vectoriales diferenciales en una variedad diferencial $M$ y
sean $X = X^i \partial / \partial x^i$ y $Y = Y^i \partial / \partial x^j$ las
expresiones de $X$ e $Y$ en terminos de unas coordenadas locales $(x^i)$ dadas
en $M$. El corchete de Lie se puede escribir entonces como:
\begin{equation*}
[X, Y] = (X^i \dfrac{\partial Y^j}{\partial x^i} - \\
Y^i \dfrac{\partial X^j}{\partial x^i}) \dfrac {\partial}{\partial x^j}
\end{equation*}
que de forma mas concisa se puede escribir como $[X, Y] = (XY^j - YX^j)
\dfrac{\partial}{\partial x^j}$.

Para todo $f, g \in C^{\infty}(M)$, $[fX, gY] = fg[X,Y] + (fXg)Y - (gYf)X$.

Una de las aplicaciones mas importantes del corchete de Lie ocurre en el
contexto de los grupos de Lie. Sea $G$ un grupo de Lie. $G$ actua suave y transitivamente sobre si misma por
traslacion por la izquierda: $L_g(h) = gh$. Un campo vectorial $X$ en $G$ se
dice que es invariante por la izquierda si es invariante bajo toda traslacion
por la izquierda. Mas explicitamente, $d(L_g)_{g'}(X_{g'}) = X_{gg'}$.

Dado que $(L_g)_{*}(aX + bY) = a(L_g)_{*}X + b(L_g)_{*}Y$, el conjunto de todos
los campos vectoriales invariantes en $G$ forman un subespacio lineal en
$\mathfrak{X}(G)$. Lo interesante de este resultado es que la propiedad de
invarianza se conserva bajo la accion del corchete de Lie.

\subsection{Algebra de Lie de un grupo de Lie}
Una algebra de Lie (sobre $\R$) es un espacio vectorial real $\g$ dotado de
una aplicacion llamada corchete de $\g \times \g$ a $\g$, normalmente
denotada por $(X, Y) \rightarrow [X, Y]$, que satisface las siguientes
propiedades:

- Bilinealidad: Para todo $a, b \in \R$,
\begin{equation*}
[aX + bY, Z] = a[X, Z] + b[Y, Z]
\end{equation*}
\begin{equation*}
[Z, aX + bY] = a[Z, X] + b[Z, Y]
\end{equation*}

- Antisimetria: $[X, Y] = -[Y, X]$

- Identidad de Jacobi: $[X, [Y, Z]] + [Y, [Z,X]] + [Z, [X, Y]] = 0$

Si $\g$ y $\h$ son algebras de Lie, una aplicacion lineal $A: \g \rightarrow \h$
se llama homomorfismo de algebras de Lie si conserva corchetes: $A[X, Y] = [AX,
AY]$. Si el homomorfismo de algebras de Lie es invertible se le llama
isomorfismo de algebras de Lie.

Un ejemplo de algebra de Lie es el espacio vectorial $M(n, \R)$ de $n
\times n$ matrices reales junto con el corchete commutador $[A, B] = AB - BA$.
Se trata de un algebra de Lie $n^2$-dimensional. Es facil comprobar las
propiedades de bilinealidad, antisimetria y la identidad de Jacobi. Este algebra
de Lie comunmente se denota por $\mathfrak{g}\mathfrak{l}(n, \R)$.

De entre todas las algebras de Lie, en la teoria de variedades diferenciales destaca el
algebra de Lie de un grupo de Lie. Esta se define como el algebra de Lie de
todos los campos vectoriales diferenciales invariantes por la izquierda sobre un
grupo de Lie $G$, que es una subalgebra de Lie de $\mathfrak{X}(G)$ y, por
tanto, un algebra de Lie. Se denota por $Lie(G)$.

De forma similar a como vemos el espacio tangente como un "modelo lineal" de una
variedad diferencial cerca de un punto, el algebra de Lie de un grupo de Lie
proporciona un "modelo lineal" de un grupo, que refleja muchas de las
propiedades del grupo. Debido a que los grupos de Lie tienen mas estructura que
las variedades diferenciales, es de esperar que sus modelos lineales tengan mas
estructura que los espacios vectoriales ordinarios. Dado que un algebra de Lie
de dimension finita es un objeto algebraico lineal, es en muchos aspectos mas
facil de entender que el mismo grupo.

Tomamos como ejemplo de un algebra de un algebra de Lie de un grupo de Lie el
espacio euclideo $\R^n$: Si consideramos $\R^n$ como un
grupo de Lie bajo la suma, la traslacion por la izquierda por un elemento $b \in
\R^n$ viene dada por la aplicacion afin $L_b(x) = b + x$, cuya diferencial
$d(L_b)$ se representa por la matriz identidad en las coordenadas estandares.
Por tanto, el campo vectorial $X^i \partial / \partial x^i$ es invariante por la
izquierda si y solo si sus coeficientes son constantes. Dado que el corchete de
Lie de dos campos vectoriales de coeficientes constantes es cero, el algebra de
Lie de $\R^n$ es abeliana e isomorfa a $\R^n$ por el corchete trivial. En
resumen, $Lie(\R^n) \cong \R^n$. El hecho de que el algebra de $\R^n$, que es un
grupo de Lie abeliano, sea un algebra de Lie abeliana no es un accidente: todo
grupo de Lie abeliano tiene un algebra de Lie abeliana. El opuesto solo es
cierto en caso de que el grupo sea conexo.

% Analizamos ahora el algebra de Lie de grupo de Lie no abeliano mas importante,
% el grupo general lineal:

% La composicion de aplicaciones naturales $Lie(GL(n, \R)) \rightarrow T_{I_n}GL(n, \R)
% \rightarrow \mathfrak{g}\mathfrak{l}(n, \R)$ determina un isomorfismo de
% algebras de Lie entre $Lie(GL(n, \R))$ y el algebra de matrices
% $\mathfrak{g}\mathfrak{l}(n, \R)$.

\subsection{Algebra de Lie de un subgrupo de Lie}
Si $G$ es un grupo de Lie y $H \subseteq G$ es un subgrupo de Lie, uno desearia
que el algebra de Lie de $H$ fuese una subalgebra de Lie de $G$. Sin embargo,
los elementos de $Lie(H)$ son campos vectoriales en $H$, no en $G$ y, por tanto,
no son elementos de $Lie(G)$, esto es, no son campos vectoriales invariantes por
la izquierda en $G$. Podemos salvar este obstaculo observando que cada elemento
de $Lie(H)$ corresponde con un unico elemento de $Lie(G)$, determinado por su
valor identidad y por la inyeccion de $Lie(H)$ en $Lie(H)$. Este isomorfismo se
caracteriza por $\mathfrak{h}=\iota_*(Lie(H)) = { X \in Lie(G): X_e \in T_eH}$,
donde $\iota_*$ denota el homomorfismo inducido de algebras de Lie.

Esta identificacion entre una subalgebra de $G$ y el algebra de Lie de $H$ se
hace visible tomando el algebra de de Lie del grupo ortogonal $O(n)$ ejemplo. El
grupo ortogonal $O(n)$ es un subgrupo de Lie de $G(n, \R)$. $O(n)$ es igual al
conjunto de nivel $\theta ^{-1}(I_n)$, donde $\theta : GL(n, \R) \rightarrow
M(n, \R)$ es la aplicacion $\theta(A) = A^TA$. El espacio tangente $T_{I_n}O(n)$
es igual al nucleo de la diferencial $d\theta_{I_n} : T_{I_n}GL(n, \R)
\rightarrow T_{I_n}M(n, \R)$. Calculando la diferencial, vemos que
$d\theta_{I_n}(B) = B^T + B$, luego $T_{I_n}O(n) = {B \in
\gl(n, \R): B^T + B = 0}$, que es el conjunto de matrices
antisimetricas de dimension $n \times n$. Denotamos este subespacio de $\gl(n,
\R)$ por $\mathfrak{o}(n)$. Por el homomorfismo inducido de algebras de Lie
expuesto en el parrafo anterior,
$\mathfrak{o}(n)$ is una subalgebra de Lie de $\gl(n, \R)$ que es isomorfo a
$Lie(O(n))$.

\section{Derivada de Lie}
La derivada de Lie nos permite calcular la velocidad de cambio de un campo
vectorial a traves del flujo de otro cambio vectorial de una forma que no
depende de las coordenadas elegidas.

\subsection{Curvas integrales y flujos}
Las curvas integrales son curvas suaves cuya velocidad en cada punto es igual al
valor del campo vectorial en ese punto. Es decir, si $X$ es un campo vectorial
en $M$, la integral de curva de $X$ es una curva diferencial $\gamma:J
\rightarrow M$ tal que $\gamma'(t)=X_{\gamma(t)}$ para todo $t\in J$. Por
ejemplo, si $X = x \dfrac{\partial}{\partial y} - y \dfrac{\partial}{\partial
x}$ y $\gamma: \R \rightarrow \R^2$ es una curva suave tal que $\gamma(t) =
(x(t), y(t))$, entonces la condicion $\gamma'(t)=X_{\gamma(t)}$ se traduce en la
ecuacion $x'(t) \dfrac{\partial}{\partial x}|_{\gamma(t)} +
y'(t)\dfrac{\partial}{\partial y}|_{\gamma(t)} = x(t) \dfrac{\partial}{\partial
y}|_{\gamma(t)} - y(t)\dfrac{\partial}{\partial x}|_{\gamma(t)}$ de soluciones $x(t)=a\cos t
- b\sin t$, $y(t)=a\sin t + b\cos t$. Luego toda curva de la
forma $\gamma(t)=(a\cos t - b\sin t, a\sin t + b\cos t)$ es una curva integral de $X$.

La coleccion de todas las curvas
integrales de un campo vectorial dado en una variedad diferencial determina una
familia de difeomorfismos de abiertos de una variedad diferencial que llamos
flujo.

\subsection{Derivada de Lie de un campo vectorial}

La motivacion en la definicion de las derivadas de Lie radica en la
dificultad de definir la derivada direccional de un campo vectorial en una
variedad diferencial cualquiera.

En el espacio euclideo, tiene sentido definir la derivada direccional de un
campo vectorial en la direccion de un vector $v \in T_p\R^n$ como:
\begin{equation*}
	D_vW(p) = \dfrac{d}{dt} \mid _{t=0} W_{p + tv} = \lim_{x\to0} \dfrac{W_{p+tv} - W_p}{t}
\end{equation*}
Se puede ver con facilidad que $D_vW(p)$ puede ser evaluado aplicando $D_v$ a
cada componente de $W$ por separado, esto es, $D_vW(p) = D_vW^i(p)
\dfrac{\partial}{\partial x^i}|_p$. Sin embargo, esta definicion depende del
hecho que $\R^n$ es un espacio vectorial, luego los vectores tangente $W_{p+tv}$
y $W_p$ pueden ser vistos como elementos de $\R^n$. Sin embargo, este
procedimiento no se puede extrapolar a una variedad diferencial arbitraria. Solo
con reemplazar $p + tv$ por una curva $\gamma(t)$ que empieza en $p$ y cuya
velocidad inicial es $v$ nos damos cuenta que $W_{\gamma(t)}$ y $W_{\gamma(0)}$
son elementos de dos espacios tangente differentes ($T_{\gamma(t)}M$ y
$T_{\gamma(0)M}$ respectivamente), luego la diferencia $W_{p+tv} - W_p$ no tiene
sentido. No hay ninguna forma que independiente de las coordenadas elegidas que
de sentido a la derivada direccional de $W$ en la direccion de un vector $v$.

El problema se soluciona si reemplazamos el vector $v \in T_pM$ con un campo
vectorial $V \in \mathfrak{X}(M)$, con lo que podemos usar el flujo de $V$ para
convertir valores de $W$ de nuevo en $p$ y asi poder operar con los distintos
elementos.

Sea $M$ una variedad diferenciable, $V$ un campo vectorial suave en $M$ y
$\theta$ el flujo de $V$. Para todo campo vectorial suave $W$ en $M$ se define
un campo vectorial en $M$, denotado por $\mathfrak{L}_VW$ que llamamos la
derivada de Lie de $W$ con respecto a $V$ y viene definida por:
\begin{equation*}
	(\mathfrak{L}_VW)_p = \dfrac{d}{dt}|_{t=0} d(\theta_{-t})_{\theta_{t}(p)}(W_{\theta_t(p)}) \\
	= \lim_{t\to0}\dfrac{d(\theta_{-t})_{\theta_{t}(p)}(W_{\theta_t(p)}) - W_p}{t}
\end{equation*}
siempre que la derivada exista. Para valores de $t \neq 0$ el limite anterior
tiene sentido: $\theta_t$ esta definida en un entorno de $p$ y $\theta_{-t}$ es
la inversa de $\theta_t$, luego ambos
$d(\theta_{-t})_{\theta_{t}(p)}(W_{\theta_t(p)})$ y $W_p$ son elementos de $T_pM$.

Al igual que vimos con la primera formula introducida en el corchete de Lie,
esta definicion de $\mathfrak{L}_VW)$ es de poca utilidad para el calculo, pues
por lo general el flujo es un objeto matematico dificil de manipular en los
calculos. Por suerte, el corchete de Lie nos proporciona una forma de hallar la
derivada de Lie sin necesidad de hallar explicitamente el flujo.

Para dos campos vectoriales suaves $V, W$ en una variedad diferenciable $M$, la
derivada de Lie de $W$ con respecto de $V$ se puede escribir como
$\mathfrak{L}_VW) = [V, W]$.

Enunciamos a continuacion algunas de las propiedades de la derivada de Lie de un
campo vectorial:

\begin{enumerate}
  \item $\mathfrak{L}_VW = - \mathfrak{L}_WV$
  \item $\mathfrak{L}_V[W, X] = [\mathfrak{L}_VW, X] + [W, \mathfrak{L}_VX]$
  \item $\mathfrak{L}_{[V, W]} = \mathfrak{L}_V \mathfrak{L}_W X -
  \mathfrak{L}_W \mathfrak{L}_V X $
  \item Si $g \in C^{\infty}(M)$, entonces $\mathfrak{L}_V(gW) = (Vg)W +
  g\mathfrak{L}_VW$
\end{enumerate}

\subsection{Derivada de Lie de un campo tensiorial}
El lenguaje de los tensores nos permite hablar de aplicaciones multilineales en
el contexto de las variedades diferenciables. En su forma mas simple, los
tensores son funciones multilineales reales de una o mas variables, como son el
producto escalar y los determinantes.

Los tensores puedes covariantes, contravariantes o mixtos. Un $k$-tensor
covariante en un espacio vectoral real de dimension finita $V$
es un elemento del producto tensorial $\underbrace{V^*\otimes...\otimes V^*}_{\text{b veces}}$, el cual
podemos ver como una funcion real multilineal de $b$ elementos de $V$; esto es
$\alpha : V \times ... \times V \rightarrow \R$. Denotamos el espacio vectorial
de todos los $b$-tensores covariantes en $V$ por
$T^{(0,b)}(V^*)=\underbrace{V^*\otimes...\otimes V^*}_{\text{b veces}}$. El
determinante, entendido como una funcion de $n$ vectores, es un ejempo de un
$n$-tensor covariante. De la misma forma, un tensor contravariante es aquel dado
por $T^{(a,0)}(V) = \underbrace{V\otimes...\otimes V}_{\text{a veces}}$ y un
tensor mixto viene definido por $T^{(a,b)}(V) = \underbrace{V\otimes...\otimes
V}_{\text{a veces}} \underbrace{V^*\otimes...\otimes V^*}_{\text{b veces}}$. Los
tensores mas importantes en la teoria de las variedades diferenciables son los
tensores covariantes.


Ahora, si en vez de un espacio vectorial real tomamos una variedad diferencial $M$,
llamamos fibrado de $k$-tensores covariantes en $M$ a $T^kT^*M = \coprod_{p \in
M}T^k(T_p^*M)$. Un campo tensorial suave es una seccion que es suave de $\pi :
T^kM \rightarrow M$. Un campo tensorial simetrico en una variedad diferencial es
simplemente un campo tensorial covariante cuyos valores en cada punto es un
tensor simetrico. Los campos tensorials alternos se llaman formas diferenciales
y los veremos mas adelante.

Antes de introducir la derivada de Lie en el contexto de campos tensoriales es
necesario definir el concepto de retroaccion o \emph{pullback}. Sea $F: M
\rightarrow N$ una aplicacion diferenciable. Para cada punto $p \in M$ y cada
$k$-tensor $\alpha \in T^k(T^*_{F(p)}N)$, definimos un tensor $dF^*_p(\alpha)
\in T^k(T^*_pM)$ dado por $dF^*_p(\alpha)(v_1,...,v_k)=
\alpha(dF_p(v_1),...,dF_p(v_k))$ para todo $v_1,...,v_k \in T_pM$. Si $A$ es un
campo de $k$-tensores en $N$, entonces el \emph{pullback} de $A$ por $F$ es otro
campo $k$-tensorial $F*A$ en $M$ y se denota por $(F^*A)_p = dF^*_p(A_{F(p)})$.
Este campo tensorial actua sobre vectores $v_1,...,v_k \in T_pM$ de esta forma:
$(F^*A)_p (v_1,...,v_k) = A_{F(p)} (dF_p(v_1),...,dF_p(v_k))$
% Definir pullback

La operacion de la derivada de Lie se puede extender a campos tensoriales de
cualquier rango. Sea $M$ una variedad diferenciable, $V$ un campo vectorial
suave en $M$ y $\theta$ su flujo. Para todo $p \in M$, si $t$ esta
suficientemente cerca de $0$, entonces $\theta_t$ es un difeomorfismo de un
entorno de $p$ a un entorno de $\theta_t(p)$, luego el \emph{pullback}
$d(\theta_t)i_p^*$ convierte tensores en $\theta_t(p)$ en tensores en $p$ por la
formula $d(\theta_t)^*_p(A_{\theta_t(p)})(v_1,..., v_k) =
A_{\theta_t(p)}(d(\theta_t)_p(v_1),..., d(\theta_t)_p(v_k)$. Podemos ver que
$d(\theta_t)^*_p(A_{\theta_t(p)})$ es el valor del \emph{pullback} del campo
tensorial $\theta^*_tA$ en $p$.

Estamos ya en disposicion de definir la derivada de Lie de un campo tensorial
$A$ de tensores covariantes en $M$ con respecto a campo vectorial $V$ como:
\begin{equation*}
	(\mathfrak{L}_VA)_p = \dfrac{d}{dt}|_{t=0}(\theta^*_tA)_p = \\
	\lim_{t\to0} \dfrac{d(\theta_t)^*_p(A_{\theta_t(p)}) - A_p}{t}
\end{equation*}

Esta expresion que derivamos pertenece a $T^k(T^*_pM)$ para todo $t$, luego
$\mathfrak{L}_VA$ tiene sentido como elemento de  $T^k(T^*_pM)$.

Destacamos sin demostrar varias propiedades de la derivada de Lie en campos de
tensores covariantes dados $M$ una variedad diferencial suave, $V \in
\mathfrak{X}(M)$, $f$ una funcion suave real (entendida como un campo
$0$-tensorial) en $M$, y $A$ y $B$ campos de tensores covariantes en $M$:

\begin{itemize}
	\item $\mathfrak{L}_V f = V f$
	\item $\mathfrak{L}_V(fA) = (\mathfrak{L}_Vf)A + f \mathfrak{L}_VA$
	\item $\mathfrak{L}_V(A \otimes B) = (\mathfrak{L}_VA) \otimes B + A \otimes
	\mathfrak{L}_VB$
	\item Si $X_1, ..., X_k$ son campos vectoriales suaves y $A$ es un campo
	$k$-tensorial tambien suave, entonces $\mathfrak{L}_V(A(X_1,...,X_k))=
	(\mathfrak{L}_VA)(X_1,...,X_k) + A(\mathfrak{L}_VX_1,..., X_k) + ... +
	A(X_1,..., \mathfrak{L}_VX_k)$
\end{itemize}

Una de las consecuencias de estas propiedades es que nos permiten expresar la
derivada de Lie de cualquier campo tensorial covariante suave en terminos del
corchete de Lie y derivadas direccionales de funciones, lo que nos vuelve a
permitir calcular derivadas de Lie sin tener que primero calcular el flujo. La
siguiente formula es de gran utilidad para los calculos:

Si $V$ es un campo vectorial suave y $A$ es campo de k-tensores covariantes
suave, entonces para cada conjunto de campos vectoriales $X_1, ..., X_k$,
\begin{equation*}
	(\mathfrak{L}_VA)(X_1,...,X_k) = V(A(X_1,..., X_k)) - A([V, X_1], X_2,..., X_k) - ... - \\
	A(X_1,..., X_{k-1}, [V, X_k])
\end{equation*}

Hacemos uso de estas formulas para calcular la derivada de Lie
en coordenadas locales $(x^i)$ de un campo de 2-tensores covariantes arbitrario.
Observamos que $\mathfrak{L}_Vdx^i=d(\mathfrak{L}_Vx^i) = d(Vx^i) = dV^i$, luego
\begin{equation*}
\begin{split}
	\mathfrak{L}_VA & = \mathfrak{L}_V(A_{ij}dx^i) \otimes dx^j \\
	& = \mathfrak{L}_V(A_{ij})dx^i \otimes dx^j + A_{ij}(\mathfrak{L}_Vdx^i) \otimes dx^j + A_{ij}dx^i \otimes (\mathfrak{L}dx^j) \\
	& = VA_{ij}dx^i \otimes dx^j + A_{ij} dV^i \otimes dx^j + A_{ij}dx^i \otimes dV^j \\
	& = (V A_{ij} + A_{kj} \dfrac{\partial V^k}{\partial x^i} + A_{ik}\dfrac{\partial V^k}{\partial x^j}) dx^i \otimes dx^j
\end{split}
\end{equation*}

Al igual que la derivada de Lie de un campo vectorial $W$ con respecto a $V$ es
cero si y solo si $W$ es invariante bajo el flujo de $V$, un campo tensorial suave $A$
en $M$ es invariante bajo un flujo $\theta$ en $M$ si para cada $t$,
$d(\theta_t)^*_p(A_{\theta_t(p)}) = A_p$.

\subsection{Derivada de Lie de una forma diferencial}
Las formas diferenciales son campos tensoriales alternos en variedades
diferenciales. Gran parte de la teoria de formas diferenciales puede verse como
una generalizacion de la teoria de los campos de vectores covariantes, que son
los ejemplos mas basicos de formas diferenciales. A los k-tensores covariantes
alternos se les llama tambien formas exteriores o k-covectores. El espacio de
todas los k-covectores se denota por $\Lambda^k(V^*)$.

Si $M$ es una variedad diferenciable, recordemos que $T^kT^*M$ es el fibrado de
k-tensores covariantes en $M$. El subconjunto de $T^kT^*M$ de los tensores
alternos se denota por $\Lambda^k T^*M$:
\begin{equation*}
	\Lambda^k T^*M = \coprod_{p\in M} \Lambda^k(T^*_pM)
\end{equation*}

Una seccion de $\Lambda^kT^*M$ se llama k-forma diferencial si es un campo
tensorial continuo cuyo valor en cada punto es un tensor alterno. El espacio
vectorial de k-formas suaves se define como $\omega^k(M) = \Gamma (\Lambda ^k
T^*M)$. El producto exterior de una k-forma con otra l-forma es una (k+l)-forma.
El operador natural de las formas diferenciales se llama derivada exterior y es
una generalizacion de la diferencial de una funcion.

Las formulas utilizadas en la seccion anterior para los campos tensioriales
tambien son aplicables a las formas diferenciales. Sin embargo, en el caso de
las formas diferenciales, la derivada exterior nos proporciona una formula
mas eficaz para calcular la derivada de Lie. Esto conlleva ademas otras
consecuencias teoricas.

La derivada de Lie satisface la regla del producto con respecto al producto
exterior. Si $M$ es una variedad diferenciable, $V \in \mathfrak{X}(M)$ y
$\omega, \eta$ son formas diferenciales en $\Omega^*(M)$, entonces
$\mathfrak{L}_V(\omega \wedge \eta) = (\mathfrak{L}_V\omega) \wedge \eta +
\omega \wedge (\mathfrak{L}_V\eta)$.

La formula mas destacada para hallar la derivada de Lie de una forma diferencial
es la llamada formula magica de Cartan:
\begin{equation*}
	\mathfrak{L}_V\omega = V \lrcorner (d\omega) + d(V \lrcorner \omega)
\end{equation*}
donde $\lrcorner$ es una aplicacion lineal $\lrcorner_v : \Lambda ^k (V^*)
\rightarrow \Lambda^{[k-1]} (V^*)$ llamada multiplicacion interior y definida como
$\lrcorner_v: \omega(w_1,...,w_{k-1}) = \omega(v, w1, ..., v_{k-1})$ para todo
$v, w$ en un espacio vectorial de dimension finita $V$.

\section{Cohomologia}

Introducimos a continuacion algunos conceptos esenciales para entender la cohomologia de De
Rham y la cohomologia del algebra de Lie.

Sea $R$ un anillo conmutativo y sea $A^* = ... \to A^{p-1} \to^d A^p \to^d A^{p+1} \to
...$ una sucesion de $R$-modulos y $R$-aplicaciones lineas. En la mayoria de
casos, el anillo sera $\Z$, en cuyo caso estaremos tratando con grupos abelianos
y homomorfismos, o $\R$, en cuyo caso tenemos espacios vectoriales y
aplicaciones lineales. La terminologia de modulos es conveniente para combinar
ambos casos. Esta sucesion se dice que es un \emph{complejo} si la composicion de dos
aplicaciones sucesivas de $d$ es la aplicacion nula $d \cdot d = 0: A^p \to
A^{p+2}$ para cada $p$. Como en el caso de los espacios vectoriales, tal
secuencia de modulos se llama exacta si la imagen de cada $d$ es igual al nucleo
de la siguiente. Claramente, toda sucesion exacta es un complejo, pero el
reciproco no es siempre cierto.

Llamamos p-esimo grupo cohomologico de una sucesion $A^*$ al cociente
\begin{equation*}
	H^p(A^*)=\dfrac{Ker(d:A^p \rightarrow A^{p+1})}{Im(d:A^{p-1}\rightarrow A^p)}
\end{equation*}
Intuitivamente, se puede ver pensar en este grupo cohomologico como una medida
de la "falta de exactitud" en $A^p$. Un ejemplo de un complejo es el complejo de
De Rham de una n-variedad diferenciable $M$:
\begin{equation*}
	0 \to \Omega^0(M)\to^d ... \to^d \Omega^p(M) \to^d \Omega^{p+1}(M) \to^d ... \to^d \Omega^n(M)\to 0
\end{equation*}

\subsection{Cohomologia de De Rham}

La cohomologia de De Rham nace del estudio de las formas cerradas y exactas. Una
forma es cerrada si su derivada $d\omega =0$; es exacta si se puede escribir
como $\omega = d\eta$. Debido a que $d \cdot d = 0$, toda forma exacta es
cerrada. Sin embargo, no toda forma cerrada es exacta. La propiedad de las formas
cerradas que tambien son exactas esta ligada a la existencia de "agujeros" en la
variedad diferenciable, que se hace medible gracias a un nuevo
conjunto de invariantes de las variedades diferenciables llamado grupo de
cohomologia de De Rham.

Un ejemplo de una forma cerrada que no es exacta en $\R^2 \backslash \{0\}$ es la
1-forma $\dfrac{xdy -ydx}{x^2+y^2}$. Sin embargo, esta forma es exacta en
dominios mas reducidos como el semiplano $H=\{(x,y):x>0\}$. Este comportamiento
es caracteristico de las formas cerradas: toda forma cerrada es siempre
localmente exacta.

Dado que $d:\Omega^p(M)\rightarrow \Omega^{p+1}(M)$ es lineal, su imagen y
nucleo son subespacios lineales. Definimos al conjunto de p-formas cerradas en $M$
como $Z^p(M)=Ker(d:\Omega^p(M \rightarrow \Omega^{p+1}(M))$ y al conjunto de p-formas
exactas como $B^p(M)=Im(d:\Omega^{p-1}(M) \rightarrow \Omega^p(M))$, tambien en
$M$.

El hecho de que toda forma exacta es cerrada implica que $B^p(M)\subseteq
Z^p(M)$. Por tanto, tiene sentido definir el grupo de cohomologia de De Rham de
grado $p$ de $M$ como el espacio vectorial cociente
$H_{dR}^p(M)=\dfrac{Z^p(M)}{B^p(M)}$. Un termino mas apropiado seria "espacio de
cohomologia de De Rham", ya que estos grupos son, en realidad, espacios
vectoriales.

Para $p<0$ o $p> \dim M$ tenemos que $H_{dR}^p(M)=0$, pues $\Omega^p(M)=0$ en esos
casos. Para $0\leq p \leq n$, la definicion implica que $H_{dR}^p(M)$ es igual a
$0$ si y solo si toda p-forma cerrada en $M$ es exacta. Por ejemplo, en el caso
anterior, el hecho de que exista una 1-forma cerrada en $\R^1\backslash\{0\}$
que no es exacta significa que $H_{dR}^1(\R^2-\{0\})$. Por otra parte, el lema
de Poincare para 1-formas implica que $H_{dR}^1(U)=0$ para todo subconjunto
$U\subseteq\R^n$ estrellado.

Los grupos de De Rham son invariantes por difeomorfismos. Para cada p-forma
cerrada $\omega$ en $M$, denotamos $[\omega]$ a la clase de equivalencia de
$\omega$ en $H_{dR}^p(M)$ llamada clase cohomologia de $\omega$. Si
$[\omega]=[\omega']$, esto es, si $\omega$ y $\omega'$ difieren por una forma
exacta, decimos que $\omega$ y $\omega'$ son cohomologicas. Dos variedades
diferenciables difeomorficas tienen grupos de cohomologia de De Rham
isomorficos.

Ademas, los grupos cohomologicos de De Rham son invariantes homotopicos: dos
variedades diferenciables homotopicamente equivalentes tienen grupos de De Rham
isomorficos. Esto se debe a que aplicaciones diferenciables homotopicas inducen
la misma aplicacion cohomologica. Una aplicacion cohomologica es una aplicacion lineal $F^*:H_{dR}^p(N)
\rightarrow H_{dR}^p(M)$, con $M$ y $N$ variedades diferenciables y $F$ una
aplicacion suave de $M$ a $N$, con las siguientes propiedades:
\begin{enumerate}
	\item Si $G : N \rightarrow P$ es otra aplicacion suave, entonces $(G \cdot
	F)^*=F^* \cdot G^* : H_{dR}^p(P) \rightarrow H_{dR}^p(M)$
	\item Si $Id$ es la aplicacion identidad en $M$, entonces $Id^*$ es la
	aplicacion identidad en $H_{dR}^p(M)$.
\end{enumerate}

Dado que todo homeomorfismo es una equivalencia homotopica, los grupos
cohomologicos de De Rham son invariantes topologicos: si dos variedades
diferenciables son homeomorfas, entonces sus grupos de cohomologia de De Rham
son isomorficos.

\subsection{Cohomologia del algebra de Lie}

La cohomologia del algebra de Lie nace del estudio de
la topologia de los grupos de Lie y los espacios homogeneos relacionando metodos
cohomologicos de De Rham a propiedades del algebra de Lie.

Si $G$ es un grupo de Lie compacto simplemente conxexo, entonces esta
determinado por su algebra de Lie, por lo que es posible calcular su cohomologia
a partir del algebra de Lie. Su cohomologia es la cohomologia de De Rham del
complejo de las formas diferenciales en $G$. Este complejo se puede reemplazar
por un complejo de formas diferenciales invariantes por la izquierda. El espacio
de las formas
invariantes por la izquierda se pueden identificar con el algebra exterior del
algebra de Lie, dotada de una diferencial apropiada.

La construccion de esta diferencial en un algebra exterior tiene sentido para
cualquier algebra de Lie, por lo que se usa para definir la cohomologia del
algebra de Lie para todas las algebras de Lie.

Si $G$ es un grupo de Lie simplemente conexo y no compacto, la cohomologia del
algebra de Lie del algebra de Lie asociada $\mathfrak{g}$ no reproduce
necesariamente la cohomologia de De Rham de $G$. De forma mas general, una
construccion similar sirve para definir la cohomologia de un algebra de Lie con
coeficientes en un modulo.

Definimos entonces la cohomologia de un algebra de Lie $\mathfrak{g}$ con coeficientes en
un $\mathfrak{g}$-modulo $M$ por la izquierda como
$H^*_{\text{Lie}}(\mathfrak{g},M)=Ext^*_{U_{\mathfrak{g}}}(k, M)$ donde $k$ es
el modulo trivial sobre el
algebra universal evolvente $U_{\mathfrak{g}}$, esto es, el algebra mas general
que contiene todas las representacion de un algebra de Lie.

Por ejemplo, para toda suma directa de algebras de
Lie simples o algebra de Lie semisimple $\mathfrak{g}$, existe una forma
de Killing $\langle-,-\rangle$. La correspondiente cadena cerrada de tres elementos (3-cociclo) es $\mu =
\langle-,[-,-]\rangle:CE(\mathfrak{g})$; es una funcion que manda tres elementos del
algebra de Lie $x$, $y$, $z$ al numero $\mu (x,y,z)=\langle x,[y,z]\rangle $.

Algunos resultados importantes de la cohomologia de las algebras de Lie incluyen
los lemas de Whitehead, el teorema de Weil y el teorema de decomposicion de
Levi.

\end{document}
